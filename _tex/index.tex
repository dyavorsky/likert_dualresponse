% Options for packages loaded elsewhere
\PassOptionsToPackage{unicode}{hyperref}
\PassOptionsToPackage{hyphens}{url}
\PassOptionsToPackage{dvipsnames,svgnames,x11names}{xcolor}
%
\documentclass[
]{article}

\usepackage{amsmath,amssymb}
\usepackage{iftex}
\ifPDFTeX
  \usepackage[T1]{fontenc}
  \usepackage[utf8]{inputenc}
  \usepackage{textcomp} % provide euro and other symbols
\else % if luatex or xetex
  \usepackage{unicode-math}
  \defaultfontfeatures{Scale=MatchLowercase}
  \defaultfontfeatures[\rmfamily]{Ligatures=TeX,Scale=1}
\fi
\usepackage{lmodern}
\ifPDFTeX\else  
    % xetex/luatex font selection
\fi
% Use upquote if available, for straight quotes in verbatim environments
\IfFileExists{upquote.sty}{\usepackage{upquote}}{}
\IfFileExists{microtype.sty}{% use microtype if available
  \usepackage[]{microtype}
  \UseMicrotypeSet[protrusion]{basicmath} % disable protrusion for tt fonts
}{}
\makeatletter
\@ifundefined{KOMAClassName}{% if non-KOMA class
  \IfFileExists{parskip.sty}{%
    \usepackage{parskip}
  }{% else
    \setlength{\parindent}{0pt}
    \setlength{\parskip}{6pt plus 2pt minus 1pt}}
}{% if KOMA class
  \KOMAoptions{parskip=half}}
\makeatother
\usepackage{xcolor}
\setlength{\emergencystretch}{3em} % prevent overfull lines
\setcounter{secnumdepth}{5}
% Make \paragraph and \subparagraph free-standing
\makeatletter
\ifx\paragraph\undefined\else
  \let\oldparagraph\paragraph
  \renewcommand{\paragraph}{
    \@ifstar
      \xxxParagraphStar
      \xxxParagraphNoStar
  }
  \newcommand{\xxxParagraphStar}[1]{\oldparagraph*{#1}\mbox{}}
  \newcommand{\xxxParagraphNoStar}[1]{\oldparagraph{#1}\mbox{}}
\fi
\ifx\subparagraph\undefined\else
  \let\oldsubparagraph\subparagraph
  \renewcommand{\subparagraph}{
    \@ifstar
      \xxxSubParagraphStar
      \xxxSubParagraphNoStar
  }
  \newcommand{\xxxSubParagraphStar}[1]{\oldsubparagraph*{#1}\mbox{}}
  \newcommand{\xxxSubParagraphNoStar}[1]{\oldsubparagraph{#1}\mbox{}}
\fi
\makeatother


\providecommand{\tightlist}{%
  \setlength{\itemsep}{0pt}\setlength{\parskip}{0pt}}\usepackage{longtable,booktabs,array}
\usepackage{calc} % for calculating minipage widths
% Correct order of tables after \paragraph or \subparagraph
\usepackage{etoolbox}
\makeatletter
\patchcmd\longtable{\par}{\if@noskipsec\mbox{}\fi\par}{}{}
\makeatother
% Allow footnotes in longtable head/foot
\IfFileExists{footnotehyper.sty}{\usepackage{footnotehyper}}{\usepackage{footnote}}
\makesavenoteenv{longtable}
\usepackage{graphicx}
\makeatletter
\def\maxwidth{\ifdim\Gin@nat@width>\linewidth\linewidth\else\Gin@nat@width\fi}
\def\maxheight{\ifdim\Gin@nat@height>\textheight\textheight\else\Gin@nat@height\fi}
\makeatother
% Scale images if necessary, so that they will not overflow the page
% margins by default, and it is still possible to overwrite the defaults
% using explicit options in \includegraphics[width, height, ...]{}
\setkeys{Gin}{width=\maxwidth,height=\maxheight,keepaspectratio}
% Set default figure placement to htbp
\makeatletter
\def\fps@figure{htbp}
\makeatother
% definitions for citeproc citations
\NewDocumentCommand\citeproctext{}{}
\NewDocumentCommand\citeproc{mm}{%
  \begingroup\def\citeproctext{#2}\cite{#1}\endgroup}
\makeatletter
 % allow citations to break across lines
 \let\@cite@ofmt\@firstofone
 % avoid brackets around text for \cite:
 \def\@biblabel#1{}
 \def\@cite#1#2{{#1\if@tempswa , #2\fi}}
\makeatother
\newlength{\cslhangindent}
\setlength{\cslhangindent}{1.5em}
\newlength{\csllabelwidth}
\setlength{\csllabelwidth}{3em}
\newenvironment{CSLReferences}[2] % #1 hanging-indent, #2 entry-spacing
 {\begin{list}{}{%
  \setlength{\itemindent}{0pt}
  \setlength{\leftmargin}{0pt}
  \setlength{\parsep}{0pt}
  % turn on hanging indent if param 1 is 1
  \ifodd #1
   \setlength{\leftmargin}{\cslhangindent}
   \setlength{\itemindent}{-1\cslhangindent}
  \fi
  % set entry spacing
  \setlength{\itemsep}{#2\baselineskip}}}
 {\end{list}}
\usepackage{calc}
\newcommand{\CSLBlock}[1]{\hfill\break\parbox[t]{\linewidth}{\strut\ignorespaces#1\strut}}
\newcommand{\CSLLeftMargin}[1]{\parbox[t]{\csllabelwidth}{\strut#1\strut}}
\newcommand{\CSLRightInline}[1]{\parbox[t]{\linewidth - \csllabelwidth}{\strut#1\strut}}
\newcommand{\CSLIndent}[1]{\hspace{\cslhangindent}#1}

\usepackage{geometry}
\usepackage{graphicx}
\usepackage{comment}

\usepackage{mathtools,amsthm,bm}
\usepackage{unicode-math}

\usepackage[usenames,dvipsnames]{xcolor}
\usepackage{hyperref}

\newcommand\mleft{\mathopen{}\mathclose\bgroup\left}
\newcommand\mright{\aftergroup\egroup\right}
\makeatletter
\@ifpackageloaded{caption}{}{\usepackage{caption}}
\AtBeginDocument{%
\ifdefined\contentsname
  \renewcommand*\contentsname{Table of contents}
\else
  \newcommand\contentsname{Table of contents}
\fi
\ifdefined\listfigurename
  \renewcommand*\listfigurename{List of Figures}
\else
  \newcommand\listfigurename{List of Figures}
\fi
\ifdefined\listtablename
  \renewcommand*\listtablename{List of Tables}
\else
  \newcommand\listtablename{List of Tables}
\fi
\ifdefined\figurename
  \renewcommand*\figurename{Figure}
\else
  \newcommand\figurename{Figure}
\fi
\ifdefined\tablename
  \renewcommand*\tablename{Table}
\else
  \newcommand\tablename{Table}
\fi
}
\@ifpackageloaded{float}{}{\usepackage{float}}
\floatstyle{ruled}
\@ifundefined{c@chapter}{\newfloat{codelisting}{h}{lop}}{\newfloat{codelisting}{h}{lop}[chapter]}
\floatname{codelisting}{Listing}
\newcommand*\listoflistings{\listof{codelisting}{List of Listings}}
\makeatother
\makeatletter
\makeatother
\makeatletter
\@ifpackageloaded{caption}{}{\usepackage{caption}}
\@ifpackageloaded{subcaption}{}{\usepackage{subcaption}}
\makeatother

\ifLuaTeX
  \usepackage{selnolig}  % disable illegal ligatures
\fi
\usepackage{bookmark}

\IfFileExists{xurl.sty}{\usepackage{xurl}}{} % add URL line breaks if available
\urlstyle{same} % disable monospaced font for URLs
\hypersetup{
  pdftitle={Likert Scale Dual Response in Conjoint Analysis},
  pdfauthor={Prachi Bhalerao; Dan Yavorsky; Geoffery Zheng},
  pdfkeywords={choice-based conjoint analysis, stated preference
elicitation, discrete choice methods, ordered choice},
  colorlinks=true,
  linkcolor={blue},
  filecolor={Maroon},
  citecolor={Blue},
  urlcolor={Blue},
  pdfcreator={LaTeX via pandoc}}


\title{Likert Scale Dual Response in Conjoint Analysis}
\author{Prachi Bhalerao \and Dan Yavorsky \and Geoffery Zheng}
\date{2024-07-02}

\begin{document}
\maketitle
\begin{abstract}
Write the abstract here \ldots{}
\end{abstract}


\section{Introduction}\label{sec-intro}

Lorem ipsum

\section{Model}\label{sec-model}

\subsection{General Specification}\label{general-specification}

Consumer \(i=1,\ldots,N\) derives utility from good
\(j \in \mathcal{J}_i =  \lbrace 0,\ldots,J_i \rbrace\) with utility
\(u_{ij}\) given by \[
u_{ij} = h(\symbfit{x}_j, \symbfit{\beta}_i) + \eta_{ij}.
\]

where \(\symbfit{x}_j\) is a vector of good characteristics,
\(\symbfit{\beta}_i\) is a vector of consumer-specific taste parameters,
and \(\eta_{ij}\) encapsulates factors known to the consumer but not to
the researcher that affect the consumer's utility and are modeled as
\(\eta_{ij} \sim \text{Gumbel} \left( 0,1 \right)\) with \(\eta_{ij}\)
assumed to be indepedent of both \(x_j\) and \(\eta_{ij'}\) for
\(j' \ne j\). We take
\(h(\symbfit{x}_j, \symbfit{\beta}_i) = \symbfit{x}_j'\symbfit{\beta}_i\)
but this specification is not required.

The ``zero''-th (or ``outside'') good is special and is associated with
a zero vector of good characteristics (\(\symbfit{x}_0 = \mathbf{0}\))
such that \(h(\symbfit{x}_0, \symbfit{\beta}_i)=0\). Consumers observe
\(\symbfit{x}_j\) and \(\eta_{ij}\) for all ``inside'' goods
(\(j>0; j \in \mathcal{J}_i\)), but they \emph{do not} observe
\(\eta_{i0}\).\footnote{This can be motivated by a framework in which
  \(u_0 = 0\) and \(u_{ij} = {x_j}'\beta_i + \eta_{ij} - \eta_0\) for
  \(j>0; j \in \mathcal{J}_i\). Here, \(\eta_0\) captures the consumer's
  uncertainty about their future tastes. Given that utilities are
  ordinal and \(\eta_0\) is a common shock, it plays no role in the
  choice among the most preferred inside good
  \(j^* \in \mathcal{J}_i\).{]}}

Consumers first report their preferred inside good among, which is given
by \[
j^*_i = \arg\max_{j>0; j \in \mathcal{J}_i} u_{ij}.
\]

Let \(u_i^*\) denote the utility of good \(j^*_i\) for consumer \(i\)
and let \(t_i = e_{j_i^*}\) indicate the ``one-hot'' encoding of the
most-preferred inside good (ie, \(t_i\) is a vector with \(J_i-1\) zeros
and \(1\) one).

Second, consumers report a value \(y_i\) on a discrete qualitative scale
\(w \in \mathcal{W} = \lbrace 1, \ldots, W \rbrace\) to indicate the
probability that they prefer good \(j^*_i\) to the outside good 0.
Consumers know \(u_i^*\), but do not know \(\eta_{i0}\), and thus this
probability is given by \[
p_i = \Pr \left( \eta_{i0} < u_i^* \right).
\]

Each consumer reports the interval \(y_i=w\) into which \(p_i\) falls \[
y_i = w \hspace{1ex} \text{s.t.} \hspace{1ex} p_i \in \left[\alpha_{w-1},\alpha_w\right).
\]

As is well known,\footnote{For example, see (McFadden 1981) and (Cardell
  1997).} \(u_i\) follows a Gumbel distribution with location parameter
\(\overline{\mu}_i\) and scale parameter \(1\), where
\begin{equation}\phantomsection\label{eq-max-mu}{
\overline{\mu}_i = \ln\left( \sum_{j\in\mathcal{J}} \exp\left({X_j}'\beta_i\right)\right).
}\end{equation}

Under these assumptions, we have \(p_i = F\left( u_i^* \right)\) with
\(F\) as the \(\text{Gumbel} \left( 0,1 \right)\) distribution.

The consumer's report of \(y_i=w\) is therefore equivalent to reporting
that \[
    u_i^* \in \left[
        F^{-1}_{\text{Gumbel}\left(0\right)}\left(\alpha_{w\left(i\right)-1}\right),
        F^{-1}_{\text{Gumbel}\left(0\right)}\left(\alpha_{w\left(i\right)}\right)
    \right),
\] where we omit the common scale parameter for brevity.

This occurs with probability \[
\begin{aligned}
    \Pr \left( u_i^* \in \mathcal{W}_{w\left( i \right)} \right) 
    &= F_{\text{Gumbel} \left( \overline{ \mu } \right) } \left( F^{-1}_{\text{Gumbel} \left( 0 \right) } \left( \alpha_{w \left( i \right)} \right) \right) - 
       F_{\text{Gumbel} \left( \overline{ \mu } \right) } \left( F^{-1}_{\text{Gumbel} \left( 0 \right) } \left( \alpha_{w \left( i \right) - 1} \right) \right) \\
    &= \left( \alpha_{w \left( i \right)} \right)^{\exp \left( \overline{\mu} \right)} - 
       \left({\alpha_{w \left( i \right)-1}} \right)^{\exp \left( \overline{\mu} \right)}.
\end{aligned}
\]

This is \(p\left(w_i\mid \alpha, \beta\right)\), the conditional
likelihood of \(w_i\).\footnote{While the parametrization of
  \(\eta_0 \sim \text{Gumbel}\left(0,1\right)\) preserves symmetry among
  the \(J+1\) goods and is thus a natural choice, the framework can
  easily accommodate an alternative distribution for \(\eta_0\). For
  example, one could use an affine function of individual
  characteristics to accommodate individual-level variation in the
  propensity to prefer the outside good.}

\subsection{Comment: Relation to the (Brazell et al. 2006) Dual Response
Model}\label{comment-relation-to-the-brazell_2006-dual-response-model}

Another common framework when soliciting consumer preferences is to
directly ask consumers if \(V_i \ge 0\). This implicitly assumes that
\(\eta_{i0} = 0\), so that consumers deterministically know whether or
not the `'inside'\,' good \(j^*_i\) is preferred to the outside good. As
we have already noted, our framework can nest that standard case by
assuming that, rather than following a standard Gumbel distribution,
\(\eta_0\) is instead a degenerate distribution. In such a case, all but
2 of the \(W\) partitions are empty, and the non-empty partitions are
\(\mathcal{W}_0 = \left\lbrace 0 \right\rbrace\) and
\(\mathcal{W}_W = \left\lbrace 1\right\rbrace\), as
\(p_i \in \left\lbrace 0, 1\right\rbrace\).

We acknowledge that this requires a slight abuse of notation, as
\(\mathcal{W}_w\) was defined above using left-closed intervals. These
have the advantage of being invertible under the inverse-CDF mapping. In
the degenerate case, we instead have
\(p\left(w_i = W\mid\alpha,\beta\right) = \mathbb{P}\left(\eta_{i0} < V_i\right) = \mathbb{P}\left(V_i > 0\right) = 1 - \exp\left(-\exp\left(\overline{\mu}\right)\right)\).

\subsection{Comment: Nesting a Single-Good
Specification}\label{comment-nesting-a-single-good-specification}

Suppose there are two goods: good \(j\) and good \(0\). For example,
suppose good \(j\) is Diet Coke and good 0 is the ``outside option'' of
not purchasing the beverage.

These goods provide utility of

\begin{itemize}
\tightlist
\item
  \(U_j = x_j'\beta + \eta_j\) and
\item
  \(U_0 = x_0'\beta + \eta_0\).
\end{itemize}

\(x\) is observed by both the consumer and the researcher (eg, the price
\(x_j\) of the Diet Coke and the price of not making a purchase
\(x_0=0\)); \(\beta\) are taste parameters known to the consumer, but
not to the researcher that are to be estimated (eg, her price
sensitivity).

\(\eta\) encapsulates factors known to the consumer but not to the
researcher that affect the consumer's utility (eg, the positive or
negative ``status'' from being observed purchasing or consuming the Diet
Coke, or from not purchasing and consuming a beverage). From the
researcher's perspective, \(\eta_j\) and \(\eta_0\) are assumed to be
independent of \(x\) and modeled as random variables with cumulative
distribution functions \(F(\eta_j)\) and \(F(\eta_0)\).

We define the difference in utility as
\(U^* = U_j - U_0 = x_j'\beta + \eta^*\) where \(x_0 = 0\) and
\(\eta^* = \eta_j - \eta_0\).

The consumer does not report \(U^*\) but rather reports \(y\), a
censoring of \(U^*\) into one of \(W\) discrete qualitative scale values
\(w=1,2,\ldots,W\). Suppose, for example, that the consumer reports the
middle level (\(y=2\)) out of three available (labeled, ``unlikely'' for
\(w=1\), ``somewhat likely'' for \(w=2\), and ``very likely'' for
\(w=3\)).

The \(W\) levels of the qualitative scale are separated at values
\(\mu_w\) such that \(-\infty < \mu_1, \mu_2, \ldots, \mu_W = \infty\),
with \(\mu_w, w=1,2,\ldots,W\) as parameters to be estimated.

Then we have that \[
\begin{aligned}
    \Pr \left( y=w \right) 
    &= \Pr \left(\mu_{w-1} < U^* < \mu_w \right) \\
    &= \Pr \left(\mu_{w-1} < x_j'\beta + \eta^* < \mu_w \right) \\
    &= \Pr \left(\mu_{w-1} - x_j'\beta < \eta^* < \mu_w - x_j'\beta \right) \\
    &= F(\mu_w - x_j'\beta) - F(\mu_{w-1} - x_j'\beta)
\end{aligned}
\]

Define \(t_w \in \lbrace 0,1 \rbrace\) as indicators with \(t_w=1\) when
\(y=w\). The individual likelihood is then

\[
\prod_{w=1}^W =  \left[ F(\mu_w - x_j'\beta) - F(\mu_{w-1} - x_j'\beta) \right]^{t_w}
\]

\subsection{Sketch of Estimation}\label{sketch-of-estimation}

\begin{itemize}
    \item Specify hyper-parameters governing the prior distribution of $\left(\symbf{\pi}, \symbf{\beta}\right)$. There is a relatively large amount of flexibility in the prior distribution over $\beta$. $\symbf{\pi} \sim \text{Dirichlet}$ deterministically maps to $\symbf{\alpha}$.
    \item (First Branch) Given draws of $\left(\alpha, \beta\right)$, the probability that consumer $i$ reports $j^*_i$ is given by a softmax
        $$
        p\left(j^*_i \mid \alpha, \beta\right) = p\left(j^*_i \mid \beta\right) = \frac{\exp\left({X_j}'\beta_i\right)}{\sum_{j'\in\mathcal{J}}\exp\left({X_{j'}}'\beta_i\right)}
        $$
    \item (Second Branch) Given $\left(\alpha, \beta\right)$, the probability that consumer $i$ reports $w_i$ is given by
    $$
        p\left(w_i \mid \alpha,\beta, j^*_i\right) = p\left(w_i\mid \alpha, \beta\right) =  \left({\alpha_{w\left(i\right)}}\right)^{\exp\left(\overline{\mu}\right)} - \left({\alpha_{w\left(i\right)-1}}\right)^{\exp\left(\overline{\mu}\right)}
    $$
    where $\overline{\mu}_i\left(\beta_i\right)$ is a function of consumer $i$'s tastes $\beta_i$ and the design matrix $X$ (@eq-max-mu). Note that the observed choice $j^*_i$ is irrelevant for this likelihood.
\end{itemize}

Thus the overall likelihood (conditional on some draw of parameters) is
\[
    p\left(\left(j^*_i, w_i\right) \mid \alpha, \beta \right) = p\left(w_i \mid \alpha,\overline{\mu}_i\left(\beta\right) \right) \times p\left(j^* \mid \beta\right).
\]

Assume some universal partitioning of the unit interval into \(W\)
disjoint intervals with cutoffs denoted by the
\(\left(W+1\right)\)-dimensional vector
\(\symbfit{\pi} \in \Delta^{W-1}\). Let \(\symbfit{\alpha}\) denote the
partial sums of \(\symbfit{\pi}\), so that \[
    \begin{array}{*9{c}}
       \symbfit{\alpha}  = \big\lbrace 0, & \pi_1, & \pi_1 + \pi_2, & \ldots, & \underbrace{\sum_{i=1}^{w} \pi_{i}}, & \ldots, & 1\big\rbrace \\
         & & & & \alpha_w 
    \end{array}
\] \[
    \vphantom{\bigcup_{w=1}^W} \mathcal{W}_w = \left[\alpha_{w-1}, \alpha_w\right)
\] \[
    \bigcup_{w=1}^W \mathcal{W}_i = \left[0, 1\right)
\]

\section{Simulation Study}\label{sec-simstudy}

add

\section{Empirical Analysis}\label{sec-analysis}

add

\section{Discussion}\label{sec-discussion}

Differs from (Brazell et al. 2006)

\section{Conclusion}\label{sec-conclusion}

add

\section*{References}\label{references}
\addcontentsline{toc}{section}{References}

\phantomsection\label{refs}
\begin{CSLReferences}{1}{0}
\bibitem[\citeproctext]{ref-Brazell_2006}
Brazell, Jeff D., Christopher G. Diener, Ekaterina Karniouchina, William
L. Moore, Válerie Séverin, and Pierre-Francois Uldry. 2006. {``The
No-Choice Option and Dual Response Choice Designs.''} \emph{Marketing
Letters} 17 (4): 255--68.
\url{https://doi.org/10.1007/s11002-006-7943-8}.

\bibitem[\citeproctext]{ref-Cardell_1997}
Cardell, N. Scott. 1997. {``Variance Components Structures for the
Extreme-Value and Logistic Distributions with Application to Models of
Heterogeneity.''} \emph{Econometric Theory} 13 (2): 185--213.
\url{https://doi.org/10.1017/s0266466600005727}.

\bibitem[\citeproctext]{ref-Mcfadden_1981}
McFadden, Daniel L. 1981. {``Structural Discrete Probability Models
Derived from Theories of Choice.''} In, 198--272. The MIT Press.

\end{CSLReferences}




\end{document}
