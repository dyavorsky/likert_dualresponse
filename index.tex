% Options for packages loaded elsewhere
\PassOptionsToPackage{unicode}{hyperref}
\PassOptionsToPackage{hyphens}{url}
\PassOptionsToPackage{dvipsnames,svgnames,x11names}{xcolor}
%
\documentclass[
  letterpaper,
  DIV=11,
  numbers=noendperiod]{scrartcl}

\usepackage{amsmath,amssymb}
\usepackage{iftex}
\ifPDFTeX
  \usepackage[T1]{fontenc}
  \usepackage[utf8]{inputenc}
  \usepackage{textcomp} % provide euro and other symbols
\else % if luatex or xetex
  \usepackage{unicode-math}
  \defaultfontfeatures{Scale=MatchLowercase}
  \defaultfontfeatures[\rmfamily]{Ligatures=TeX,Scale=1}
\fi
\usepackage{lmodern}
\ifPDFTeX\else  
    % xetex/luatex font selection
\fi
% Use upquote if available, for straight quotes in verbatim environments
\IfFileExists{upquote.sty}{\usepackage{upquote}}{}
\IfFileExists{microtype.sty}{% use microtype if available
  \usepackage[]{microtype}
  \UseMicrotypeSet[protrusion]{basicmath} % disable protrusion for tt fonts
}{}
\makeatletter
\@ifundefined{KOMAClassName}{% if non-KOMA class
  \IfFileExists{parskip.sty}{%
    \usepackage{parskip}
  }{% else
    \setlength{\parindent}{0pt}
    \setlength{\parskip}{6pt plus 2pt minus 1pt}}
}{% if KOMA class
  \KOMAoptions{parskip=half}}
\makeatother
\usepackage{xcolor}
\setlength{\emergencystretch}{3em} % prevent overfull lines
\setcounter{secnumdepth}{5}
% Make \paragraph and \subparagraph free-standing
\ifx\paragraph\undefined\else
  \let\oldparagraph\paragraph
  \renewcommand{\paragraph}[1]{\oldparagraph{#1}\mbox{}}
\fi
\ifx\subparagraph\undefined\else
  \let\oldsubparagraph\subparagraph
  \renewcommand{\subparagraph}[1]{\oldsubparagraph{#1}\mbox{}}
\fi


\providecommand{\tightlist}{%
  \setlength{\itemsep}{0pt}\setlength{\parskip}{0pt}}\usepackage{longtable,booktabs,array}
\usepackage{calc} % for calculating minipage widths
% Correct order of tables after \paragraph or \subparagraph
\usepackage{etoolbox}
\makeatletter
\patchcmd\longtable{\par}{\if@noskipsec\mbox{}\fi\par}{}{}
\makeatother
% Allow footnotes in longtable head/foot
\IfFileExists{footnotehyper.sty}{\usepackage{footnotehyper}}{\usepackage{footnote}}
\makesavenoteenv{longtable}
\usepackage{graphicx}
\makeatletter
\def\maxwidth{\ifdim\Gin@nat@width>\linewidth\linewidth\else\Gin@nat@width\fi}
\def\maxheight{\ifdim\Gin@nat@height>\textheight\textheight\else\Gin@nat@height\fi}
\makeatother
% Scale images if necessary, so that they will not overflow the page
% margins by default, and it is still possible to overwrite the defaults
% using explicit options in \includegraphics[width, height, ...]{}
\setkeys{Gin}{width=\maxwidth,height=\maxheight,keepaspectratio}
% Set default figure placement to htbp
\makeatletter
\def\fps@figure{htbp}
\makeatother
% definitions for citeproc citations
\NewDocumentCommand\citeproctext{}{}
\NewDocumentCommand\citeproc{mm}{%
  \begingroup\def\citeproctext{#2}\cite{#1}\endgroup}
\makeatletter
 % allow citations to break across lines
 \let\@cite@ofmt\@firstofone
 % avoid brackets around text for \cite:
 \def\@biblabel#1{}
 \def\@cite#1#2{{#1\if@tempswa , #2\fi}}
\makeatother
\newlength{\cslhangindent}
\setlength{\cslhangindent}{1.5em}
\newlength{\csllabelwidth}
\setlength{\csllabelwidth}{3em}
\newenvironment{CSLReferences}[2] % #1 hanging-indent, #2 entry-spacing
 {\begin{list}{}{%
  \setlength{\itemindent}{0pt}
  \setlength{\leftmargin}{0pt}
  \setlength{\parsep}{0pt}
  % turn on hanging indent if param 1 is 1
  \ifodd #1
   \setlength{\leftmargin}{\cslhangindent}
   \setlength{\itemindent}{-1\cslhangindent}
  \fi
  % set entry spacing
  \setlength{\itemsep}{#2\baselineskip}}}
 {\end{list}}
\usepackage{calc}
\newcommand{\CSLBlock}[1]{\hfill\break\parbox[t]{\linewidth}{\strut\ignorespaces#1\strut}}
\newcommand{\CSLLeftMargin}[1]{\parbox[t]{\csllabelwidth}{\strut#1\strut}}
\newcommand{\CSLRightInline}[1]{\parbox[t]{\linewidth - \csllabelwidth}{\strut#1\strut}}
\newcommand{\CSLIndent}[1]{\hspace{\cslhangindent}#1}

\usepackage{geometry}
\usepackage{graphicx}
\usepackage{comment}

\usepackage{mathtools,amsthm,mleftright,bm}
\usepackage{unicode-math}

\usepackage[usenames,dvipsnames]{xcolor}
\usepackage{hyperref}
\KOMAoption{captions}{tableheading}
\makeatletter
\@ifpackageloaded{caption}{}{\usepackage{caption}}
\AtBeginDocument{%
\ifdefined\contentsname
  \renewcommand*\contentsname{Table of contents}
\else
  \newcommand\contentsname{Table of contents}
\fi
\ifdefined\listfigurename
  \renewcommand*\listfigurename{List of Figures}
\else
  \newcommand\listfigurename{List of Figures}
\fi
\ifdefined\listtablename
  \renewcommand*\listtablename{List of Tables}
\else
  \newcommand\listtablename{List of Tables}
\fi
\ifdefined\figurename
  \renewcommand*\figurename{Figure}
\else
  \newcommand\figurename{Figure}
\fi
\ifdefined\tablename
  \renewcommand*\tablename{Table}
\else
  \newcommand\tablename{Table}
\fi
}
\@ifpackageloaded{float}{}{\usepackage{float}}
\floatstyle{ruled}
\@ifundefined{c@chapter}{\newfloat{codelisting}{h}{lop}}{\newfloat{codelisting}{h}{lop}[chapter]}
\floatname{codelisting}{Listing}
\newcommand*\listoflistings{\listof{codelisting}{List of Listings}}
\makeatother
\makeatletter
\makeatother
\makeatletter
\@ifpackageloaded{caption}{}{\usepackage{caption}}
\@ifpackageloaded{subcaption}{}{\usepackage{subcaption}}
\makeatother
\ifLuaTeX
  \usepackage{selnolig}  % disable illegal ligatures
\fi
\usepackage{bookmark}

\IfFileExists{xurl.sty}{\usepackage{xurl}}{} % add URL line breaks if available
\urlstyle{same} % disable monospaced font for URLs
\hypersetup{
  pdftitle={Likert Scale Dual Response in Conjoint Analysis},
  pdfauthor={Prachi Bhalerao; Dan Yavorsky; Geoffery Zheng},
  pdfkeywords={choice-based conjoint analysis, stated preference
elicitation, discrete choice methods},
  colorlinks=true,
  linkcolor={blue},
  filecolor={Maroon},
  citecolor={Blue},
  urlcolor={Blue},
  pdfcreator={LaTeX via pandoc}}

\title{Likert Scale Dual Response in Conjoint Analysis}
\author{Prachi Bhalerao \and Dan Yavorsky \and Geoffery Zheng}
\date{2024-07-01}

\begin{document}
\maketitle
\begin{abstract}
Write the abstract here \ldots{}
\end{abstract}

\section{Introduction}\label{sec-intro}

\begin{verbatim}
[1] -0.8475514
\end{verbatim}

\textsubscript{Source:
\href{https://dyavorsky.github.io/likert_dualresponse/index.qmd.html}{Article
Notebook}}

\section{Model}\label{sec-model}

Consumers are indexed by \(i=1,2,\ldots\,I\). Goods are indexed by
\(j=1,2,\ldots\,J\) and each good is characterized by a vector of
characteristics \(x_j\). Each consumer has: (i) taste parameters
\(\beta_i\), (ii) a good-specific taste shock \(\varepsilon_{ij}\), and
(iii) a good-specific taste shock \(\nu_{ij}\). The outside good is
indexed with \(j=0\) and \(x_j = \mathbf{0}\).

Consumer \(i\) derives utility \(U_{ij}\) from good \(j\), where

\begin{equation}\phantomsection\label{eq-utility}{
U_{ij} = {x_j}' \beta_{i} + \nu_{ij} + \varepsilon_{ij}.
}\end{equation}

For each consumer \(i\), the econometrician observes:

\begin{itemize}
\tightlist
\item
  The most preferred good
  \(j \in \left\lbrace 1, 2, \ldots\,J\right\rbrace\), denoted
  \(j_i^{(1)}\). Let \(t_i = e_{j^{(1)}}\) represent the ``one-hot'\,'
  encoding of consumer \(i\)'s choice of good \(j^{(1)}\);
\item
  A signal related to the probability that the consumer prefers the
  outside good \(j=0\). This signal takes the form of an ordinal scale
  \(w \in \left\lbrace 1, 2, \ldots, W \right\rbrace\). Higher \(w_i\)
  corresponds to increased probability of choosing \(j_i^{(1)}\) over
  the outside good \(j=0\).
\end{itemize}

In addition, the econometrician observes the matrix of good
characteristics \(\symbfit{x}\).

\subsection{Special Case of the
Log-Likelihood}\label{special-case-of-the-log-likelihood}

Assume - The \(\nu\) taste shocks are independent and identically
distributed Gumbel-distributed with location parameter 0 and scale
parameter 1, - The \(\varepsilon\) taste shocks are all zero, and -
\(W=2\).\footnote{As will be explained, this is without loss of generality, as the intermediate partitions would all be empty.}

Given these assumptions, it is well-known that the probability that
\(t_i = e_j\) (meaning consumer \(i\) prefers good \(j\) over all other
goods in \(\mathcal{J}\)) is given by

\[
p\mleft(t \mid \beta_i,x_j\mright) = \frac{\prod_{j\in{}\mathcal{J}} U\mleft(x_j;\;\beta_i\mright)^{t_j}}{\sum_{j\in{}\mathcal{J}} U\mleft(x_j;\;\beta_i\mright) }
\] \{ex\_addname\}

where

\[
U\mleft(x;\;\beta\mright) = \exp\left\lbrace x'\beta\right\rbrace.
\]

Conditional on \(\symbfit{\beta}\), a \(I\times{}J\) matrix of consumer
taste parameters, the likelihood function is

\[
p\mleft(\left(\symbfit{t},\symbfit{w}\right)\mid \symbfit{x},\symbfit{\beta}\mright) = \prod_{i\in{}\mathcal{I}} p\mleft(\left(t_i,w_i\right) \mid \beta_i,x_j\mright)
\]

where \[
\begin{align}
    \prod_{i\in{}\mathcal{I}} p\mleft(\left(t_i,w_i\right) \mid \beta_i,x_j\mright) &= \prod_{i\in\mathcal{I}} p\mleft(t_i\mid\beta_i,x_j\mright) p\mleft(w_i\mid\beta_i,x_j,t_i\mright) \\
    &= \prod_{i\in\mathcal{I}} p\mleft(t_i\mid\beta_i,x_j\mright) \int_{\mathbb{R}^J} p\mleft({t_i}'\left(\symbfit{x}'\beta_i+\nu_i\right)\in \mathcal{W}_i \mid \beta_i,x_j,t_i\mright) \mathrm{d}F_\nu\mleft(\nu_{i}\mright) \\
    &= \prod_{i\in\mathcal{I}} p\mleft(t_i\mid\beta_i,x_j\mright) \int_{\mathbb{R}^J} p\mleft({t_i}'\left(\symbfit{x}'\beta_i+\nu_i\right)\in \mathcal{W}_i \mid \beta_i,x_j,t_i\mright) \mathrm{d}F_\nu\mleft(\nu_{i}\mright)
    %\prod_{i\in{}\mathcal{I}} \frac{\prod_{j\in{}\mathcal{J}} U\mleft(x_j;\;\beta_i\mright)^{t_j}}{1 + \sum_{j\in{}\mathcal{J}} U\mleft(x_j;\;\beta_i\mright) } \\
    %&= \prod_{i\in{}\mathcal{I}} \frac{\prod_{j\in{}\mathcal{J}} U\mleft(x_j;\;\beta_i\mright)^{t_j}}{\sum_{j\in{}\mathcal{J}} U\mleft(x_j;\;\beta_i\mright) } \, \frac{\sum_{j\in{}\mathcal{J}} U\mleft(x_j;\;\beta_i\mright) }{1 + \sum_{j\in{}\mathcal{J}} U\mleft(x_j;\;\beta_i\mright) } \\
    %&= \prod_{i\in{}\mathcal{I}} p\mleft(t_i \mid \beta_i,x_j\mright) \, \left(1 - F\mleft(-t_i \cdot \left({x_j}'\beta_i\right)\mright)\right)
\end{align}
\]

where \(F\) is a Gumbel distribution that depends on the number of goods
\(J\). Note that, when \(\varepsilon = \symbfit{0}\), \(w\) can only be
interpreted as \texttt{buy\textquotesingle{}\textquotesingle{}\ and}not
buy'\,'.\footnote{Here, the consumer has no uncertainty regarding whether or not she prefers to buy or not buy, so $\mathcal{W}_1$ contains the ``not buy'' responses and $\mathcal{W}_W$ contains the ``buy'' responses. All intermediate bins ought to be empty. Hence, without loss of generality, we can consider $W=2$.}
Here, \(\mathcal{W}_i\) denotes the \(i\)-th partition of
\(\mathbb{R}\). Since \(\nu=0\), this becomes

\[
\begin{align}
    p\mleft({t_i}'\left(\symbfit{x}'\beta_i+\nu_i\right)\in{}\mathcal{W}_1 \mid \beta_i,x_j,t_i\mright) &= \mathbf{1}_{{t_i}'\left(\symbfit{x}'\beta_i+\nu_i\right) \le 0}\\
    p\mleft({t_i}'\left(\symbfit{x}'\beta_i+\nu_i\right)\in \mathcal{W}_2 \mid \beta_i,x_j,t_i\mright) &= \mathbf{1}_{{t_i}'\left(\symbfit{x}'\beta_i+\nu_i\right) > 0}.
\end{align}
\]

Integrating over the latent \(\nu_{i}\) recovers the classic discrete
choice model

\[
F_\nu\mleft({t_i}'\left(\symbfit{x}'\beta_i+\nu_i\right)\mright)
\]

\section{Simulation Study}\label{sec-simstudy}

add

\section{Empirical Analysis}\label{sec-analysis}

add

\section{Discussion}\label{sec-discussion}

add

\section{Conclusion}\label{sec-conclusion}

add

\section*{References}\label{references}
\addcontentsline{toc}{section}{References}

\phantomsection\label{refs}
\begin{CSLReferences}{0}{1}
\end{CSLReferences}



\end{document}
